\documentclass[dvipsnames,hidelinks,t]{beamer}

  % Enables the use of colour.
  \usepackage{xcolor}
  % Syntax high-lighting for code. Requires Python's pygments.
  \usepackage{minted}
  % Enables the use of umlauts and other accents.
  \usepackage[utf8]{inputenc}
  % Diagrams.
  \usepackage{tikz}
  % Settings for captions, such as sideways captions.
  \usepackage{caption}
  % Symbols for units, like degrees and ohms.
  \usepackage{gensymb}
  % Latin modern fonts - better looking than the defaults.
  \usepackage{lmodern}
  % Allows for columns spanning multiple rows in tables.
  \usepackage{multirow}
  % Better looking tables, including nicer borders.
  \usepackage{booktabs}
  % More math symbols.
  \usepackage{amssymb}
  % More math fonts, like mathbb.
  \usepackage{amsfonts}
  % More math layouts, equation arrays, etc.
  \usepackage{amsmath}
  % More theorem environments.
  \usepackage{amsthm}
  % More column formats for tables.
  \usepackage{array}
  % Adjust the sizes of box environments.
  \usepackage{adjustbox}
  % Better looking single quotes in verbatim and minted environments.
  \usepackage{upquote}
  % Better blank space decisions.
  \usepackage{xspace}
  % Better looking tikz trees.
  \usepackage{forest}
  % URLs.
  \usepackage{hyperref}
  % For plotting.
  \usepackage{pgfplots}
  
  % Various tikz libraries.
  % For drawing mind maps.
  \usetikzlibrary{mindmap}
  % For adding shadows.
  \usetikzlibrary{shadows}
  % Extra arrows tips.
  \usetikzlibrary{arrows.meta}
  % Old arrows.
  \usetikzlibrary{arrows}
  % Automata.
  \usetikzlibrary{automata}
  % For more positioning options.
  \usetikzlibrary{positioning}
  % Creating chains of nodes on a line.
  \usetikzlibrary{chains}
  % Fitting node to contain set of coordinates.
  \usetikzlibrary{fit}
  % Extra shapes for drawing.
  \usetikzlibrary{shapes}
  % For markings on paths.
  \usetikzlibrary{decorations.markings}
  % For advanced calculations.
  \usetikzlibrary{calc}
  
  % GMIT colours.
  \definecolor{gmitblue}{RGB}{20,134,225}
  \definecolor{gmitred}{RGB}{220,20,60}
  \definecolor{gmitgrey}{RGB}{67,67,67}
  
  % Change some style options.
  \usetheme{metropolis}
  % Tell minted to use the following colour scheme. 
  \usemintedstyle{manni}
  % Remove some of the vertical space after the title.
  % \addtobeamertemplate{frametitle}{}{\vspace{-3mm}}
  % Change the default theme colours.
  \setbeamercolor{normal text}{fg=darkgray, bg=white}
  \setbeamercolor{alerted text}{fg=gmitred, bg=white}
  \setbeamercolor{example text}{fg=gmitblue, bg=white}
  \setbeamercolor{frametitle}{fg=gmitblue, bg=white}
  \setbeamercolor*{item}{fg=gmitblue}
  % Use a better math mode font.
  \usefonttheme[onlymath]{serif}
  % Don't display section pages.
  \metroset{sectionpage=none}
  % Change the default itemize bullets.
  \setbeamertemplate{itemize item}{\color{gray}--}
  % Change the position of left aligned math.
  %\setlength{\mathindent}{7mm}

  % An environment for displaying math in red, without lots of vertical space.
  \newcommand{\redmath}[1]{\vspace{-3mm} {\begin{center} \color{gmitred} \( #1 \) \end{center}} \vspace{-2mm}}
  \newcommand{\red}[1]{\vspace{-3mm} {\begin{center} \color{gmitred} #1 \end{center}} \vspace{-2mm}}

  % For displaying a blank character.
  \newcommand{\bl}{\underline{\hspace{2mm}}}

  % \citeurl can be used to a clickable short url to a slide as a reference.
  \renewcommand\footnoterule{}
  \newcommand{\citeurl}[1]{\let\thefootnote\relax\footnotetext{\tiny \textcolor{gmitgrey}{\href{http://#1}{#1}}}}
  \newcommand{\citeeg}[1]{\let\thefootnote\relax\footnotetext{\tiny \textcolor{gmitgrey}{#1}}}
  
  % Prevent minted from showing errors.
  \makeatletter
  \expandafter\def\csname PYGdefault@tok@err\endcsname{\def\PYGdefault@bc##1{{\strut ##1}}}
  \makeatother
  
  \begin{document}
    \title{Decision problems}
    \subtitle{}
    \author{ian.mcloughlin@gmit.ie}
    \date{}
  
    \begin{frame}
      \titlepage
    \end{frame}
  
    \begin{frame}{Decision problem}
  \redmath{f: S \rightarrow T \textrm{ where } |T| = 2 }

    A decision problem is a map to a set with two elements.

  \vspace{4mm}
  \begin{exampleblock}{Example}
    \[ f: \{0,1\}^* \rightarrow \{0,1\} \]
    \[ f(s) = 0 \Leftrightarrow |s| \equiv_2 0 \]
  \end{exampleblock}

  \vspace{4mm}
  \begin{exampleblock}{Another example}
   \[ f: \{0,1\}^* \rightarrow \{0,1\} \]
   \[ f(s) = 0 \Leftrightarrow wt(s) \equiv_2 0 \]
  \end{exampleblock}
\end{frame}


\begin{frame}{Recap on Languages}
  \begin{description}
    \setlength\itemsep{6mm}
    \item[Alphabet:] finite set of symbols, denoted \( \Sigma \).
    \item[String:] tuple \( w \) over \( \Sigma \).
    \item[Star:] all strings over \( \Sigma \), denoted \( \Sigma^* \).
    \item[Language:] subset \( L \) of \( \Sigma^* \).
    \item[Length:] of a string, denoted \( |w| \).
  \end{description}
\end{frame}


\begin{frame}{Turing machines recap}
  For a given input a Turing machine does one of three things:
  \vspace{4mm}
  \begin{description}
    \setlength\itemsep{6mm}
    \item[Accepts] the input string by finishing in the accept state in a finite number of steps.
    \item[Rejects] the input string by finishing in the reject/fail state in a finite number of steps.
    \item[Continues] indefinitely in some sort of infinite loop.
  \end{description}
  \vspace{4mm}
  Remember there are a finite number of states and tape symbols.
\end{frame}


\begin{frame}{Deciders}
  \redmath{f: \Sigma^* \rightarrow \{ q_f, q_a \}}
  \vspace{2mm}
  \begin{description}
    \setlength\itemsep{6mm}
    \item[Decider:] a Turing machine that always finishes in a finite number of steps.
    \item[Decides:] decides the language it accepts.
    \item[Decidable:] a language is called decidable if any Turing machine decides it.
  \end{description}
  \vspace{2mm}
  \red{Important question: are all languages decidable?}
\end{frame}



\begin{frame}{PRIMES}
  Is there a decider that decides if the input is a prime number?

  \red{Yes}

  We can make a Turing machine that counts up to a number, divides and checks the remainder is zero.

  \vspace{2mm}

  We call the language containing strings representing the prime numbers in binary form PRIMES.

  \red{PRIMES \( = \{ x \mid x \in \mathbb{N} \textrm{ and } x \textrm{ is prime} \} \) }

  So, PRIMES is decidable.
\end{frame}



\begin{frame}{An undecidable language}
  
  \begin{itemize}
    \setlength\itemsep{6mm}
    \item Encode all Turing machines as strings over some finite alphabet.
    \item Consider the subset of Turing machines that are deciders.
    \item This set is undecidable.
  \end{itemize}

\end{frame} 
  \end{document}