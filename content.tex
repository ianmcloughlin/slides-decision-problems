\begin{frame}{Decision problem}
  \redmath{f: S \rightarrow T \textrm{ where } |T| = 2 }

    A decision problem is a map to a set with two elements.

  \vspace{4mm}
  \begin{exampleblock}{Example}
    \[ f: \{0,1\}^* \rightarrow \{0,1\} \]
    \[ f(s) = 0 \Leftrightarrow |s| \equiv_2 0 \]
  \end{exampleblock}

  \vspace{4mm}
  \begin{exampleblock}{Another example}
   \[ f: \{0,1\}^* \rightarrow \{0,1\} \]
   \[ f(s) = 0 \Leftrightarrow wt(s) \equiv_2 0 \]
  \end{exampleblock}
\end{frame}


\begin{frame}{Recap on Languages}
  \begin{description}
    \setlength\itemsep{6mm}
    \item[Alphabet:] finite set of symbols, denoted \( \Sigma \).
    \item[String:] tuple \( w \) over \( \Sigma \).
    \item[Star:] all strings over \( \Sigma \), denoted \( \Sigma^* \).
    \item[Language:] subset \( L \) of \( \Sigma^* \).
    \item[Length:] of a string, denoted \( |w| \).
  \end{description}
\end{frame}


\begin{frame}{Turing machines recap}
  For a given input a Turing machine does one of three things:
  \vspace{4mm}
  \begin{description}
    \setlength\itemsep{6mm}
    \item[Accepts] the input string by finishing in the accept state in a finite number of steps.
    \item[Rejects] the input string by finishing in the reject/fail state in a finite number of steps.
    \item[Continues] indefinitely in some sort of infinite loop.
  \end{description}
  \vspace{4mm}
  Remember there are a finite number of states and tape symbols.
\end{frame}


\begin{frame}{Deciders}
  \redmath{f: \Sigma^* \rightarrow \{ q_f, q_a \}}
  \vspace{2mm}
  \begin{description}
    \setlength\itemsep{6mm}
    \item[Decider:] a Turing machine that always finishes in a finite number of steps.
    \item[Decides:] decides the language it accepts.
    \item[Decidable:] a language is called decidable if any Turing machine decides it.
  \end{description}
  \vspace{2mm}
  \red{Important question: are all languages decidable?}
\end{frame}



\begin{frame}{PRIMES}
  Is there a decider that decides if the input is a prime number?

  \red{Yes}

  We can make a Turing machine that counts up to a number, divides and checks the remainder is zero.

  \vspace{2mm}

  We call the language containing strings representing the prime numbers in binary form PRIMES.

  \red{PRIMES \( = \{ x \mid x \in \mathbb{N} \textrm{ and } x \textrm{ is prime} \} \) }

  So, PRIMES is decidable.
\end{frame}



\begin{frame}{An undecidable language}
  
  \begin{itemize}
    \setlength\itemsep{6mm}
    \item Encode all Turing machines as strings over some finite alphabet.
    \item Consider the subset of Turing machines that are deciders.
    \item This set is undecidable.
    \item Closely related to the halting problem.
  \end{itemize}

\end{frame}